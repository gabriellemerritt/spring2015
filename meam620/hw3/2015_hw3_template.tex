\documentclass[11pt,english]{article}

%\usepackage[latin9]{inputenc}
\usepackage[letterpaper]{geometry}
\geometry{verbose,tmargin=1in,bmargin=1in,lmargin=1in,rmargin=1in}
\usepackage{babel}
\usepackage{amsmath}
\usepackage{amssymb}
\usepackage{graphicx}

\newcommand{\rthree}{\mathbb{R}^3}
\title{MEAM 620 Advanced Robotics: Homework 3\\
Due: Wednesday, March 4, 2015,  11:59am }
 \author{}
 
\date{}

\begin{document}
\maketitle
\begin{enumerate}
\item
(20pts) We have defined an edge point as a point where the gradient magnitude of the image $\| \nabla I \| $ reaches a local maximum along the gradient direction. This means that the derivative of $\| \nabla I \| $ along the gradient direction $\frac{\nabla I}{\| \nabla I \|}$ has a zero crossing. Compute 
\[
\nabla_\eta \| \nabla I \|
\qquad \mbox{where} \qquad  \eta = \frac{\nabla I}{\| \nabla I \|} .
\]
 You have to know (look up) how to differentiate the magnitude of a vector $\| v \|$ with respect to the vector $v$. 

\item
(80pts)
Let $M$ be the autocorrelation matrix of a corner detector 
\[
M = \sum_{(x,y)\in \mathcal{N}(x_0,y_0)}  \begin{pmatrix}
I_x(x,y)^2 & I_x(x,y) I_y(x,y) \\ I_x(x,y) I_y(x,y) & I_y(x,y)^2\end{pmatrix}  .
\]
a. What will happen to the trace of the matrix if the image will be dilated $I'(x,y) = I(x/2,y/2)$.
Assume that $I_x,I_y$ are the image derivatives directly (without any Gaussian convolution) and that the neighborhood of summation is double the original size. 
\\
b. What will happen to the trace of the matrix if the image will be rotated by 45$\deg$ ?
\\
c. Compute the eigenvalues of the matrix if the neighborhood contains only one straight edge at 45 degrees orientation:
\[
I(x,y) = \left\{
	\begin{array}{ll}
		1  & \mbox{if } x+y \geq 0 \\
		0 & \mbox{if } x+y < 0
	\end{array}
\right.
\]
d. In this last question we want to see whether the big red rectangle is a better Harris corner than the small one. 
% We will assume that the image reads
% \begin{figure}[htb]
% \centerline{\includegraphics[width=1in]{whiteRed.png}}
% \end{figure}
\[
I(x,y) = \left\{
	\begin{array}{ll}
		1  & \mbox{if } x^2+y^2  \leq r\\
		0 & \mbox{if } \mbox{otherwise}
	\end{array}
\right.
\]
yielding a  gradient in the direction of the radius $\nabla i = (\cos\theta,\sin\theta)$.
The large rectangle extends for $\theta=0..\frac{\pi}{4}$ while the small rectangle extends for  $\theta=0..\frac{\pi}{8}$. Compute the autocorrelation matrix in both cases by replacing the sum with an integral, i.e., compute $\int\int \frac{\partial{I}}{\partial{x}}\,\,dxdy$, etc. Compute in both cases the trace and the determinant. Which of the rectangle interiors has more ``cornerness'' ?



\end{enumerate}

\end{document}
