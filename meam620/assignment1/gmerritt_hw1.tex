%
\documentclass[english]{article}
%%%%%%%%%%%%%%%%%%%%%%%%%%%%%%%%%%%%%%%%%%%%%%%%%%%%%%%%%%%%%%%%%%%%%%%%%%%%%%%%%%%%%%%%%%%%%%%%%%%%%%%%%%%%%%%%%%%%%%%%%%%%
\usepackage{latexsym,amsmath,amssymb,amsfonts,fullpage}


\begin{document}
Gabrielle Merritt 
\\
gmerritt@seas.upenn.edu 
\begin{center}
{\textbf{MEAM 620 Homework 1}} \\
Due: Monday, January 26, 11:59pm 
\end{center}






\paragraph{1.}

Show (or disprove) that the $A^*$ algorithm reduces (is equivalent in terms of computations) to: 
(a) Dijkstra's algorithm if $h=0$; and (b) the depth-first search algorithm if $h$ is the depth of the graph. 
\paragraph{solution} 
Dijkstra's algorithm can be expressed as 
$$ h(y) = d(x,y) $$ where d(x,y) is the node with minimum distance from current node
A* can be expressed in the same terms 
$h(x) -->0 $
$$ h(y) = d(x,y) + h(x) $$
$$ h(y) = d(x,y) $$ 
So when heuristic function is 0, the next node can be expressed the same way as Dijkstra.
\\ 
If h is the size of the graph, the A* algorithm does not produce a Depth first search algorithm. Instead produces a breadth first search we must use 1/h in order to change A* into a  depth first search. 

\paragraph{2.}

(a) Show that the set of rotation matrices forms a group. (b) Is the set of orthogonal matrices a group? Explain. 
\paragraph{Solution} 
a) A group must show : closure, associativity, share an Identity, and an inverse. For rotation matrices we can show that: 
$$R^T*R = R*R^T = I $$ 
$$R^T*R = R^T*R$$
$$ R^{-1} = R^{T} $$

Because Rotation matrices are orthogonal the transpose times the original matrix is equal to the identity which proves closure over the set ( two rotation matrices multiplied together form another rotation matrix) , and the common identity. In the second equation  we can see the rotation matrices are associative. Finally in the third equation we can see that there exists an inverse. 
\\
b) The equations above are properties of all orthogonal matrices of which rotation matrices are a special set. So we can also assume that a set of orthogonal matrices can be considered a group. 


\paragraph{3.}

Write down the appropriate number of  independent constraints on the 9 elements of a rotation matrix: 
\[
{\bf R} = \begin{bmatrix} R_{11} & R_{12} & R_{13}  \\
R_{21} & R_{22} & R_{23}  \\
R_{31} & R_{32} & R_{33} \end{bmatrix}
\]
\paragraph{Solution}
Since the matrix is a rotation matrix we can assume that 
$$R^{T}*R = I$$ 
from which we get the 6 equations which constrain the matrix 
 $$\begin{bmatrix} R_{11} & R_{12} & R_{13}  \\
R_{21} & R_{22} & R_{23}  \\
R_{31} & R_{32} & R_{33} \end{bmatrix}
 *  \begin{bmatrix} R_{11} & R_{21} & R_{31}  \\
R_{12} & R_{22} & R_{32}  \\
R_{13} & R_{23} & R_{33} \end{bmatrix} = \begin{bmatrix}
1 & 0 & 0\\
0 & 1 & 0 \\ 
0& 0 & 1\end{bmatrix}
$$
$$
R{11}*R{11} + R{12}*R{12}+ R{13}*R{13} = 1 $$
$$
R{11}*R{21} + R{12}*R{22}+ R{13}*R{23} = 0 $$
$$
R{11}*R{31} + R{12}*R{23}+ R{13}*R{33} = 0$$ 
$$
R{21}*R{21} + R{22}*R{22}+ R{23}*R{23} = 1$$
$$ 
R{21}*R{31} + R{22}*R{32}+ R{23}*R{33} = 0 $$
$$
 R{31}*R{31} + R{32}*R{32}+ R{33}*R{33} = 1 $$



\paragraph{4.} 

For a rotation matrix as defined in Problem 3, prove that: 
\[
R_{11} = \begin{vmatrix}
R_{22} & R_{23}  \\
R_{32} & R_{33} \end{vmatrix}
\] 
\paragraph{Solution}
If we rewrite $R$ as : 
$$
x_0\cdot x_1= \begin{vmatrix}
y_0\cdot y_1 &z_1\cdot y_0   \\
y_1\cdot z_0 &z_0\cdot z_1  \end{vmatrix}
$$
where using Binet- Cauchy Identity 
$$(y1 \cdot y_0 \bullet z_1 \cdot z_0) - (z_1 \cdot y_0 \bullet y_1 \cdot z_0 )  = (y_0 \times z_0)  \bullet (y_1 \times z_1) $$

$$ x_0 \cdot x_1 =  (y_0 \times z_0)  \bullet (y_1 \times z_1) $$


\paragraph{5.}

Consider a $2 \times 2$ rotation matrix which takes the form
\[
{\bf R} = \begin{bmatrix} \cos \theta & -\sin \theta  \\
\sin \theta  & \cos \theta \end{bmatrix}, 
\]
and another fixed but arbitrarily chosen $2 \times 2$ rotation matrix {\bf S}. 
Show that ${\bf R}^{-1}$ is a continuous function of ${\bf R}$ and {\bf RS} is a continuous function of {\bf R}.  
\paragraph{Solution}
For all $ \epsilon  > 0 \exists \delta$ such that 
$$
|| R_{\theta}  - R_{\phi} || < \delta  = || R_{\theta}^{-1} - R_{\phi}^{-1} || < \epsilon  
$$
Where the norm is a Frobenious norm 
$$ 
 R = \sum \begin{bmatrix} (\cos \theta)^2  & (-\sin \theta )^2 \\
(\sin \theta)^2  & (\cos \theta)^2 \end{bmatrix}   -    \sum \begin{bmatrix} (\cos \phi)^2  & (-\sin \phi^2 \\
(\sin \phi)^2  & (\cos \phi)^2 \end{bmatrix}< \delta
$$

$$ 
R = R^T 
$$
Since the norm will square the negative value we get 
$$ 
||R_{\theta} - R_{\phi} ||  = ||R^T_{\theta} - R^T_{\phi}|| 
$$ 
so we must pick a $ \delta \leq \epsilon $ 

b) 
For a fixed matrix S 
$$ 
|| R^T_{\theta} - R^T_{\phi} ||  = ||R_\theta *S - R_\phi *S  || 
$$ 
this is the same as above except we must chose a  $ \delta \leq \epsilon/S $

\paragraph{6.}

Starting from the second phase of project 1, you will need to model the dynamics of a quadrotor.
The equations of motion (EOMs) are dependent on the orientation of the quadrotor.
It is convenient to use a three-dimensional parameterization for the rotation. We will use use Euler angles.
However, the direct use of Euler angles is not preferred when deriving the EOMs because they have singularities. Instead, we use rotation matrices. 
There are several ways to define the Euler angles. 
In our convention, we rotate along $Z-X-Y$ axes respectively to move from world frame to the body frame. 
In other words, first rotation is along the world-$Z$ axis; then along $X$ and the $Y$ axes respectively. 
Find the rotation matrix which corresponds to $(yaw, roll, pitch) = (\psi,\phi,\theta) = (30, 15, 5)$ degrees of rotation along the axes $Z-X-Y$ respectively.

\paragraph{Solution}
$$
R_{Z,X,Y} = \begin{bmatrix}( \cos \psi *\cos \theta  -\sin \phi * \sin \psi * \sin \theta) & -\cos \phi* \sin \psi& ( cos\phi * \sin \theta + cos \ theta * sin \ phi * \sin \phi ) \\ 
(\cos \theta * \sin \psi + \cos psi * \sin *phi * \sin \theta) & \cos \phi * \cos \psi &( \sin psi * \sin \theta - \cos \psi* \cos \theta* \sin \phi) \\ 
- \cos \phi *\sin \theta & \sin \phi &  \cos \phi * \cos \theta 
\end{bmatrix}
$$

$$ 
R_{Z,X,Y} = \begin{bmatrix}
-.5725 & -0.7506  & -0.3302 \\ 
 -0.3765 &  -0.1172  & 0.9190 \\ 
  -0.7285&  0.6503 & -0.2155 
\end{bmatrix}
$$

\end{document}
